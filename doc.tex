\documentclass[12pt]{jarticle}
\usepackage[utf8]{inputenc}
\usepackage[top=30truemm, bottom=30truemm, left=10truemm, right=10truemm]{geometry}
\usepackage{amsmath}
\usepackage{amssymb}
\usepackage{mathtools}
\usepackage{siunitx}
\usepackage{bm}
\usepackage[dvipdfmx]{graphicx}
\usepackage[dvipdfmx]{color}
\usepackage[subrefformat=parens]{subcaption}
\usepackage{booktabs}
\usepackage{tabularx}
\usepackage{tocloft}
\usepackage{enumerate}
\usepackage{url}
\usepackage{multirow}
\usepackage{caption}
\usepackage{enumitem}
\usepackage{makecell}
\usepackage{array}
\usepackage{algorithm}
\usepackage{algpseudocode}
\usepackage{physics}
\usepackage{hyperref}
\usepackage{dsfont}

\numberwithin{equation}{section}    % 数式番号にセクション番号をつける
\numberwithin{figure}{section}      % 図番号にセクション
\numberwithin{table}{section}      % 図番号にセクション

% \renewcommand{\figurename}{Fig.}    % 図 -> Fig.
% \renewcommand{\tablename}{Table }   % 表 -> Table

\renewcommand{\baselinestretch}{1.1}

\newlength{\figcaptionskip}
\setlength{\figcaptionskip}{5pt} % 図のキャプション間隔
\newlength{\tabcaptionskip}
\setlength{\tabcaptionskip}{-5pt} % 表のキャプション間隔
\captionsetup[figure]{skip=\figcaptionskip}
\captionsetup[table]{skip=\tabcaptionskip}

\setlist[enumerate]{topsep=5pt, partopsep=5pt, itemsep=0pt, parsep=0pt}
\setlength{\topsep}{5pt}
\setlength{\partopsep}{5pt}

\DeclareSIUnit{\fps}{fps}
\DeclareSIUnit{\dBFS}{dBFS}

\captionsetup[subfigure]{labelformat=simple}
\renewcommand{\thesubfigure}{(\alph{subfigure})}

\allowdisplaybreaks[4]

\begin{document}

\section{はじめに}

本資料では、ランキング問題におけるオフ方策評価を扱う。

\section{推薦枠内におけるランキングのオフ方策評価}

\subsection{KPIを設定する}

KPIは、推薦枠において発生する総CV数とする。

\subsection{データの観測構造をモデル化する}

対象となる推薦枠においてコンバージョンが発生するまでの一連の流れを以下のように整理する。

\begin{enumerate}
    \item ユーザーがサービスに訪問する。
    \item 訪問ユーザーに対して、推薦枠内に複数のアイテムがランキング形式で表示される。
    \item ユーザーの興味において、推薦枠内のいずれかのアイテムがクリックされる。クリックが発生しないケースもあり得る。
    \item クリックされたアイテムについて、アイテム詳細ページに遷移した後、ユーザーの意思決定に基づきコンバージョンが発生するか否かが決定される。
\end{enumerate}

ログデータ$\mathcal{D}$は、以下のように定義する。
\begin{equation}
    \mathcal{D} = \{(u_{i}, S_{i}, C_{i}, C_{i}R_{i})\}_{i = 1}^{n}
\end{equation}
ここで、$u_{i}$はユーザー、$S_{i}$はランキング形式で表示されたアイテム、$C_{i}$は$S_{i}$の各アイテムがクリックされたかを表すベクトル、$C_{i}R_{i}$は$S_{i}$の各アイテムがCVされたかを表すベクトルである。

ログデータ$\mathcal{D}$の各要素は、独立同分布に従うとする。ログデータ$\mathcal{D}$の従う確率分布は、
\begin{equation}
    p(\mathcal{D}) = \prod_{i = 1}^{n} p(u_{i}, S_{i}, C_{i}, C_{i}R_{i}) = \prod_{i = 1}^{n} p(u_{i}) \pi_{0}(S_{i} | x_{u_{i}}) p(C_{i}, R_{i} | x_{u_{i}}, S_{i})
\end{equation}
とする。

\subsection{解くべき問題を特定する}

KPIは、推薦枠において発生する総CV数とした。これより、方策の価値は推薦枠における期待CV数を表す
\begin{equation}
    V(\pi) = \mathbb{E}_{p(u)\pi(S|x_{u})p(C, R|x_{u}, S)}\left[\sum_{k = 1}^{K} C(k)R(k)\right]
\end{equation}
とする。

\subsection{観測データのみを用いて問題を解く方法を考える}

既存の推薦モデル$\pi_{0}$の下で収集されたログデータ$D_{0}$のみを用いて、新たな推薦モデル$\pi$を導入した場合の方策の価値$V(\pi)$を推定するための推定量を考える。また、各推定量について、方策の価値$V(\pi)$に対するバイアスとバリアンスを導出し、推定量の性質を考察する。

\subsubsection{IPS推定量}

IPS推定量は、
\begin{equation}
    \hat{V}_{\text{IPS}}(\pi; \mathcal{D}) = \frac{1}{n} \sum_{i = 1}^{n}\frac{\pi(S_{i}|x_{u_{i}})}{\pi_{0}(S_{i}|x_{u_{i}})} \sum_{k = 1}^{K}C_{i}(k)R_{i}(k)
\end{equation}
で定義される推定量である。

まず、IPS推定量のバイアスを求めるため、$p(\mathcal{D})$の下での期待値を求める。
\begin{align}
    \mathbb{E}_{p(\mathcal{D})}\left[\hat{V}_{\text{IPS}}(\pi; \mathcal{D})\right] & = \mathbb{E}_{p(\mathcal{D})}\left[\frac{1}{n} \sum_{i = 1}^{n}\frac{\pi(S_{i}|x_{u_{i}})}{\pi_{0}(S_{i}|x_{u_{i}})} \sum_{k = 1}^{K}C_{i}(k)R_{i}(k)\right]                                                          \\
                                                                                   & = \frac{1}{n} \sum_{i = 1}^{n} \mathbb{E}_{p(\mathcal{D})}\left[\frac{\pi(S_{i}|x_{u_{i}})}{\pi_{0}(S_{i}|x_{u_{i}})} \sum_{k = 1}^{K}C_{i}(k)R_{i}(k)\right]                                                         \\
                                                                                   & = \frac{1}{n} \sum_{i = 1}^{n} \mathbb{E}_{p(u_{i}) \pi_{0}(S_{i} | x_{u_{i}}) p(C_{i}, R_{i} | x_{u_{i}}, S_{i})}\left[\frac{\pi(S_{i}|x_{u_{i}})}{\pi_{0}(S_{i}|x_{u_{i}})} \sum_{k = 1}^{K}C_{i}(k)R_{i}(k)\right] \\
                                                                                   & = \mathbb{E}_{p(u) \pi_{0}(S | x_{u}) p(C, R | x_{u}, S)}\left[\frac{\pi(S|x_{u})}{\pi_{0}(S|x_{u})} \sum_{k = 1}^{K}C(k)R(k)\right]                                                                                  \\
                                                                                   & = \mathbb{E}_{p(u)\pi_{0}(S|x_{u})}\left[\mathbb{E}_{p(C, R|x_{u}, S)}\left[\frac{\pi(S|x_{u})}{\pi_{0}(S|x_{u})} \sum_{k = 1}^{K}C(k)R(k)\right]\right]                                                              \\
                                                                                   & = \mathbb{E}_{p(u)\pi_{0}(S|x_{u})}\left[\frac{\pi(S|x_{u})}{\pi_{0}(S|x_{u})} \mathbb{E}_{p(C, R|x_{u}, S)}\left[\sum_{k = 1}^{K}C(k)R(k)\right]\right]                                                              \\
                                                                                   & = \mathbb{E}_{p(u)}\left[\mathbb{E}_{\pi_{0}(S|x_{u})}\left[\frac{\pi(S|x_{u})}{\pi_{0}(S|x_{u})} \mathbb{E}_{p(C, R|x_{u}, S)}\left[\sum_{k = 1}^{K}C(k)R(k)\right]\right]\right]                                    \\
                                                                                   & = \mathbb{E}_{p(u)}\left[\sum_{S \in \mathcal{S}} \pi_{0}(S|x_{u}) \frac{\pi(S|x_{u})}{\pi_{0}(S|x_{u})} \mathbb{E}_{p(C, R|x_{u}, S)}\left[\sum_{k = 1}^{K}C(k)R(k)\right]\right]                                    \\
                                                                                   & = \mathbb{E}_{p(u)}\left[\sum_{S \in \mathcal{S}} \pi(S|x_{u}) \mathbb{E}_{p(C, R|x_{u}, S)}\left[\sum_{k = 1}^{K}C(k)R(k)\right]\right]                                                                              \\
                                                                                   & = \mathbb{E}_{p(u)}\left[\mathbb{E}_{\pi(S|x_{u})}\left[\mathbb{E}_{p(C, R|x_{u}, S)}\left[\sum_{k = 1}^{K}C(k)R(k)\right]\right]\right]                                                                              \\
                                                                                   & = \mathbb{E}_{p(u)\pi(S|x_{u})p(C,R|x_{u},S)}\left[\sum_{k = 1}^{K}C(k)R(k)\right]                                                                                                                                    \\
                                                                                   & = V(\pi)
\end{align}
これより、IPS推定量のバイアスは0である。

次に、IPS推定量のバリアンスを求める。
\begin{align}
    V_{p(\mathcal{D})}\left[\hat{V}_{\text{IPS}}(\pi; \mathcal{D})\right]
     & = V_{p(\mathcal{D})}\left[\frac{1}{n} \sum_{i = 1}^{n}\frac{\pi(S_{i}|x_{u_{i}})}{\pi_{0}(S_{i}|x_{u_{i}})} \sum_{k = 1}^{K}C_{i}(k)R_{i}(k)\right]                                                                 \\
     & = \frac{1}{n^{2}} \sum_{i = 1}^{n} V_{p(\mathcal{D})}\left[\frac{\pi(S_{i}|x_{u_{i}})}{\pi_{0}(S_{i}|x_{u_{i}})} \sum_{k = 1}^{K}C_{i}(k)R_{i}(k)\right]                                                            \\
     & = \frac{1}{n} V_{p(u)\pi_{0}(S|x_{u})p(C,R|x_{u},S)}\left[\frac{\pi(S|x_{u})}{\pi_{0}(S|x_{u})} \sum_{k = 1}^{K}C(k)R(k)\right]                                                                                     \\
     & = \frac{1}{n} \begin{aligned}[t]
                         \Biggl( & \mathbb{E}_{p(u)\pi_{0}(S|x_{u})}\left[V_{p(C,R|x_{u},S)}\left[\frac{\pi(S|x_{u})}{\pi_{0}(S|x_{u})} \sum_{k = 1}^{K}C(k)R(k)\right]\right]           \\
                                 & + V_{p(u)\pi_{0}(S|x_{u})}\left[\mathbb{E}_{p(C,R|x_{u},S)}\left[\frac{\pi(S|x_{u})}{\pi_{0}(S|x_{u})} \sum_{k = 1}^{K}C(k)R(k)\right]\right] \Biggr)
                     \end{aligned}                                       \\
     & = \frac{1}{n} \begin{aligned}[t]
                         \Biggl( & \mathbb{E}_{p(u)\pi_{0}(S|x_{u})}\left[\left(\frac{\pi(S|x_{u})}{\pi_{0}(S|x_{u})}\right)^{2}V_{p(C,R|x_{u},S)}\left[\sum_{k = 1}^{K}C(k)R(k)\right]\right] \\
                                 & + V_{p(u)\pi_{0}(S|x_{u})}\left[\frac{\pi(S|x_{u})}{\pi_{0}(S|x_{u})}\mathbb{E}_{p(C,R|x_{u},S)}\left[\sum_{k = 1}^{K}C(k)R(k)\right]\right] \Biggr)
                     \end{aligned}
\end{align}
アイテム数が多い場合にはランキングの組み合わせも多くなるため、$\pi_{0}(S|x_{u})$が小さくなることでバリアンスが大きくなることが予想される。

\subsubsection{SNIPS推定量}

SNIPS推定量は、
\begin{equation}
    \hat{V}_{\text{SNIPS}}(\pi; \mathcal{D}) = \frac{1}{\sum_{i = 1}^{n} \frac{\pi(S_{i}|x_{u_{i}})}{\pi_{0}(S_{i}|x_{u_{i}})}} \sum_{i = 1}^{n}\frac{\pi(S_{i}|x_{u_{i}})}{\pi_{0}(S_{i}|x_{u_{i}})} \sum_{k = 1}^{K}C_{i}(k)R_{i}(k)
\end{equation}
で定義される推定量である。

IPS推定量と異なるためバイアスは0にならないが、重要度重みが正規化されることでバリアンスを低減できることが期待される。

\subsubsection{Clipped IPS推定量}

Clipped IPS推定量は、
\begin{equation}
    \hat{V}_{\text{cIPS}}(\pi; \mathcal{D}, M) = \frac{1}{n} \sum_{i = 1}^{n} \min\left\{\frac{\pi(S_{i}|x_{u_{i}})}{\pi_{0}(S_{i}|x_{u_{i}})}, M\right\} \sum_{k = 1}^{K}C_{i}(k)R_{i}(k)
\end{equation}
で定義される推定量である。

まず、Clipped IPS推定量のバイアスは、
\begin{align}
    \text{Bias}\left[\hat{V}_{\text{cIPS}}(\pi; \mathcal{D}, M)\right] & = \mathbb{E}_{p(\mathcal{D})}\left[\hat{V}_{\text{cIPS}}(\pi; \mathcal{D}, M)\right] - V(\pi)                                                                                                                                                            \\
                                                                       & = \begin{aligned}[t]
                                                                               \mathbb{E}_{p(\mathcal{D})}\left[\frac{1}{n} \sum_{i = 1}^{n} \min\left\{\frac{\pi(S_{i}|x_{u_{i}})}{\pi_{0}(S_{i}|x_{u_{i}})}, M\right\} \sum_{k = 1}^{K}C_{i}(k)R_{i}(k)\right] \\
                                                                               - \mathbb{E}_{p(\mathcal{D})}\left[\frac{1}{n} \sum_{i = 1}^{n}\frac{\pi(S_{i}|x_{u_{i}})}{\pi_{0}(S_{i}|x_{u_{i}})} \sum_{k = 1}^{K}C_{i}(k)R_{i}(k)\right]
                                                                           \end{aligned}                                                                    \\
                                                                       & = \mathbb{E}_{p(\mathcal{D})}\left[\frac{1}{n} \sum_{i = 1}^{n} \left(\min\left\{\frac{\pi(S_{i}|x_{u_{i}})}{\pi_{0}(S_{i}|x_{u_{i}})}, M\right\} - \frac{\pi(S_{i}|x_{u_{i}})}{\pi_{0}(S_{i}|x_{u_{i}})}\right) \sum_{k = 1}^{K}C_{i}(k)R_{i}(k)\right]
\end{align}
となる。ここでは、$V(\pi) = \mathbb{E}_{p(\mathcal{D})}\left[\hat{V}_{\text{IPS}}(\pi; \mathcal{D})\right]$であることを用いた。
\begin{equation}
    \left(\min\left\{\frac{\pi(S_{i}|x_{u_{i}})}{\pi_{0}(S_{i}|x_{u_{i}})}, M\right\} - \frac{\pi(S_{i}|x_{u_{i}})}{\pi_{0}(S_{i}|x_{u_{i}})}\right) \le 0
\end{equation}
であるから、Clipped IPS推定量は不偏推定量ではなく、バイアスは0以下の値をとることがわかる。

次に、Clipped IPS推定量のバリアンスは、IPS推定量と同様にして
\begin{align}
    V_{p(\mathcal{D})}\left[\hat{V}_{\text{cIPS}}(\pi; \mathcal{D}, M)\right]
     & = V_{p(\mathcal{D})}\left[\frac{1}{n} \sum_{i = 1}^{n}\min\left\{\frac{\pi(S_{i}|x_{u_{i}})}{\pi_{0}(S_{i}|x_{u_{i}})}, M\right\} \sum_{k = 1}^{K}C_{i}(k)R_{i}(k)\right]      \\
     & = \frac{1}{n^{2}} \sum_{i = 1}^{n} V_{p(\mathcal{D})}\left[\min\left\{\frac{\pi(S_{i}|x_{u_{i}})}{\pi_{0}(S_{i}|x_{u_{i}})}, M\right\} \sum_{k = 1}^{K}C_{i}(k)R_{i}(k)\right] \\
     & = \frac{1}{n} V_{p(u)\pi_{0}(S|x_{u})p(C,R|x_{u},S)}\left[\min\left\{\frac{\pi(S|x_{u})}{\pi_{0}(S|x_{u})}, M\right\} \sum_{k = 1}^{K}C(k)R(k)\right]
\end{align}
となる。Mで抑えることで、IPS推定量よりもバリアンスが小さくなることがわかる。

\subsubsection{IIPS推定量}

IIPS推定量は、
\begin{equation}
    \hat{V}_{\text{IIPS}}(\pi;\mathcal{D}) = \frac{1}{n}\sum_{i = 1}^{n}\sum_{k = 1}^{K} \frac{\pi(S_{i}(k) | x_{u_{i}}, k)}{\pi_{0}(S_{i}(k) | x_{u_{i}}, k)} C_{i}(k)R_{i}(k)
\end{equation}
で定義される推定量である。

まず、IIPS推定量のバイアスを求めるため、$p(\mathcal{D})$の下での期待値を求める。
\begin{align}
    \mathbb{E}_{p(\mathcal{D})}\left[\hat{V}_{\text{IIPS}}(\pi;\mathcal{D})\right] & = \mathbb{E}_{p(\mathcal{D})}\left[\frac{1}{n}\sum_{i = 1}^{n}\sum_{k = 1}^{K} \frac{\pi(S_{i}(k) | x_{u_{i}}, k)}{\pi_{0}(S_{i}(k) | x_{u_{i}}, k)} C_{i}(k)R_{i}(k)\right]                                                                \\
                                                                                   & = \frac{1}{n} \sum_{i = 1}^{n} \mathbb{E}_{p(u)\pi_{0}(S|x_{u})p(C,R|x_{u},S)} \left[\sum_{k = 1}^{K} \frac{\pi(S(k) | x_{u}, k)}{\pi_{0}(S(k) | x_{u}, k)} C(k)R(k)\right]                                                                 \\
                                                                                   & = \mathbb{E}_{p(u)}\left[\mathbb{E}_{\pi_{0}(S|x_{u})}\left[\mathbb{E}_{p(C,R|x_{u},S)}\left[\sum_{k = 1}^{K} \frac{\pi(S(k) | x_{u}, k)}{\pi_{0}(S(k) | x_{u}, k)} C(k)R(k)\right]\right]\right]                                           \\
                                                                                   & = \mathbb{E}_{p(u)}\left[\mathbb{E}_{\pi_{0}(S|x_{u})}\left[\sum_{k = 1}^{K} \frac{\pi(S(k) | x_{u}, k)}{\pi_{0}(S(k) | x_{u}, k)} \mathbb{E}_{p(C,R|x_{u},S)}\left[C(k)R(k)\right]\right]\right]                                           \\
                                                                                   & = \mathbb{E}_{p(u)}\left[\mathbb{E}_{\pi_{0}(S|x_{u})}\left[\sum_{k = 1}^{K} \frac{\pi(S(k) | x_{u}, k)}{\pi_{0}(S(k) | x_{u}, k)} \mathbb{E}_{p(C,R|x_{u},S(k))}\left[C(k)R(k)\right]\right]\right]                                        \\
                                                                                   & = \mathbb{E}_{p(u)}\left[\sum_{S \in \mathcal{S}} \pi_{0}(S|x_{u}) \sum_{k = 1}^{K} \frac{\pi(S(k) | x_{u}, k)}{\pi_{0}(S(k) | x_{u}, k)} \mathbb{E}_{p(C,R|x_{u},S(k))}\left[C(k)R(k)\right]\right]                                        \\
                                                                                   & = \mathbb{E}_{p(u)}\left[\sum_{S \in \mathcal{S}} \pi_{0}(S|x_{u}) \sum_{k = 1}^{K} \sum_{a \in \mathcal{A}} \frac{\pi(a | x_{u}, k)}{\pi_{0}(a | x_{u}, k)} \mathbb{E}_{p(C,R|x_{u},a)}\left[C(k)R(k)\right] \mathds{1}\{S(k) = a\}\right] \\
                                                                                   & = \mathbb{E}_{p(u)}\left[\sum_{k = 1}^{K} \sum_{a \in \mathcal{A}} \frac{\pi(a | x_{u}, k)}{\pi_{0}(a | x_{u}, k)} \mathbb{E}_{p(C,R|x_{u},a)}\left[C(k)R(k)\right] \sum_{S \in \mathcal{S}} \pi_{0}(S|x_{u}) \mathds{1}\{S(k) = a\}\right] \\
                                                                                   & = \mathbb{E}_{p(u)}\left[\sum_{k = 1}^{K} \sum_{a \in \mathcal{A}} \frac{\pi(a | x_{u}, k)}{\pi_{0}(a | x_{u}, k)} \mathbb{E}_{p(C,R|x_{u},a)}\left[C(k)R(k)\right] \pi_{0}(a|x_{u},k)\right]                                               \\
                                                                                   & = \mathbb{E}_{p(u)}\left[\sum_{k = 1}^{K} \sum_{a \in \mathcal{A}} \pi(a | x_{u}, k) \mathbb{E}_{p(C,R|x_{u},a)}\left[C(k)R(k)\right]\right]                                                                                                \\
                                                                                   & = \mathbb{E}_{p(u)}\left[\sum_{k = 1}^{K} \sum_{a \in \mathcal{A}} \mathbb{E}_{p(C,R|x_{u},a)}\left[C(k)R(k)\right] \left(\sum_{S \in \mathcal{S}} \pi(S|x_{u}) \mathds{1}\{S(k) = a\}\right)\right]                                        \\
                                                                                   & = \mathbb{E}_{p(u)}\left[\sum_{k = 1}^{K} \sum_{a \in \mathcal{A}} \sum_{S \in \mathcal{S}} \pi(S|x_{u}) \mathbb{E}_{p(C,R|x_{u},a)}\left[C(k)R(k)\right] \mathds{1}\{S(k) = a\}\right]                                                     \\
                                                                                   & = \mathbb{E}_{p(u)}\left[\sum_{S \in \mathcal{S}} \pi(S|x_{u}) \sum_{k = 1}^{K} \sum_{a \in \mathcal{A}} \mathbb{E}_{p(C,R|x_{u},a)}\left[C(k)R(k)\right] \mathds{1}\{S(k) = a\}\right]                                                     \\
                                                                                   & = \mathbb{E}_{p(u)\pi(S|x_{u})}\left[\sum_{k = 1}^{K} \mathbb{E}_{p(C,R|x_{u},S(k))}\left[C(k)R(k)\right]\right]                                                                                                                            \\
                                                                                   & =\mathbb{E}_{p(u)\pi(S|x_{u})}\left[\sum_{k = 1}^{K} \mathbb{E}_{p(C,R|x_{u},S)}\left[C(k)R(k)\right]\right]                                                                                                                                \\
                                                                                   & =\mathbb{E}_{p(u)\pi(S|x_{u})p(C,R|x_{u},S)}\left[\sum_{k = 1}^{K} C(k)R(k)\right]                                                                                                                                                          \\
                                                                                   & = V(\pi)
\end{align}
これより、IIPS推定量のバイアスは、独立性が満たされる状況では0である。

次に、IIPS推定量のバリアンスを求める。
\begin{align}
    V_{p(\mathcal{D})}\left[\hat{V}_{\text{IIPS}}(\pi;\mathcal{D})\right] & = V_{p(\mathcal{D})}\left[\frac{1}{n}\sum_{i = 1}^{n}\sum_{k = 1}^{K} \frac{\pi(S_{i}(k) | x_{u_{i}}, k)}{\pi_{0}(S_{i}(k) | x_{u_{i}}, k)} C_{i}(k)R_{i}(k)\right]                                                                                                                                  \\
                                                                          & = \frac{1}{n} V_{p(u)\pi_{0}(S|x_{u})p(C,R|x_{u},S)}\left[\sum_{k = 1}^{K} \frac{\pi(S_{i}(k) | x_{u_{i}}, k)}{\pi_{0}(S_{i}(k) | x_{u_{i}}, k)} C_{i}(k)R_{i}(k)\right]                                                                                                                             \\
                                                                          & = \frac{1}{n} \begin{aligned}[t]
                                                                                              \Biggl( & \mathbb{E}_{p(u)\pi_{0}(S|x_{u})}\left[V_{p(C,R|x_{u},S)}\left[\sum_{k = 1}^{K} \frac{\pi(S_{i}(k) | x_{u_{i}}, k)}{\pi_{0}(S_{i}(k) | x_{u_{i}}, k)} C_{i}(k)R_{i}(k)\right]\right]           \\
                                                                                                      & + V_{p(u)\pi_{0}(S|x_{u})}\left[\mathbb{E}_{p(C,R|x_{u},S)}\left[\sum_{k = 1}^{K} \frac{\pi(S_{i}(k) | x_{u_{i}}, k)}{\pi_{0}(S_{i}(k) | x_{u_{i}}, k)} C_{i}(k)R_{i}(k)\right]\right] \Biggr)
                                                                                          \end{aligned} \\
\end{align}
IPS推定量のような方策のランキングに対する確率ではなく、アイテムに対する確率になっていることから、IPS推定量よりもバリアンスが小さくなることを期待できる。

\subsubsection{SNIIPS推定量}

SNIIPS推定量は、
\begin{equation}
    \hat{V}_{\text{SNIIPS}}(\pi;\mathcal{D}) = \sum_{i = 1}^{n}\sum_{k = 1}^{K} \frac{\frac{\pi(S_{i}(k) | x_{u_{i}}, k)}{\pi_{0}(S_{i}(k) | x_{u_{i}}, k)}}{\sum_{j = 1}^{n} \frac{\pi(S_{j}(k) | x_{u_{j}}, k)}{\pi_{0}(S_{j}(k) | x_{u_{j}}, k)}} C_{i}(k)R_{i}(k)
\end{equation}
で定義される推定量である。

IIPS推定量と異なるためバイアスは0にならないが、重要度重みが正規化されることでバリアンスを低減できることが期待される。

\subsubsection{Naive推定量}

Naive推定量は、
\begin{equation}
    \hat{V}_{\text{Naive}}(\pi;\mathcal{D}) = \frac{1}{\sum_{i = 1}^{n}\sum_{k = 1}^{K} \pi(S_{i}(k)|x_{u_{i}}, k)} \sum_{i = 1}^{n}\sum_{k = 1}^{K} \pi(S_{i}(k)|x_{u_{i}}, k) C_{i}(k)R_{i}(k)
\end{equation}
で定義される推定量である。

これは、$\pi$が確定的な方策の場合を考えると、ログを収集した時の方策$\pi_{0}$が出したランキングのk番目のアイテムと、新たな方策$\pi$が出したランキングのk番目のアイテムが一致した時のみ、その報酬を加算して平均した値となる。実際、i番目のサンプルに対して$\pi$がランキング$T_{i}$を確定的に出すとすると、
\begin{align}
    \hat{V}_{\text{Naive}}(\pi;\mathcal{D}) & = \frac{1}{\sum_{i = 1}^{n}\sum_{k = 1}^{K} \pi(S_{i}(k)|x_{u_{i}}, k)} \sum_{i = 1}^{n}\sum_{k = 1}^{K} \pi(S_{i}(k)|x_{u_{i}}, k) C_{i}(k)R_{i}(k)               \\
                                            & = \frac{1}{\sum_{i = 1}^{n}\sum_{k = 1}^{K} \mathds{1}\{S_{i}(k) = T_{i}(k)\}} \sum_{i = 1}^{n}\sum_{k = 1}^{K} \mathds{1}\{S_{i}(k) = T_{i}(k)\} C_{i}(k)R_{i}(k)
\end{align}

バイアスはあるが、バリアンスを低減する効果があるようである。

\subsubsection{DM推定量}

DM推定量は、
\begin{equation}
    \hat{V}_{\text{DM}}(\pi;\mathcal{D}) = \frac{1}{n} \sum_{i = 1}^{n} \sum_{k = 1}^{K} \mathbb{E}_{\pi(S|x_{u_{i}})}\left[\hat{q}(x_{u_{i}}, e_{S(k)})\right]
\end{equation}
で定義される推定量である。ここで、
\begin{equation}
    q(x_{u_{i}}, e_{S(k)}) = \mathbb{E}\left[C(k)R(k) | x_{u}, e_{S(k)}\right]
\end{equation}
である。すなわち、$q(x_{u_{i}}, e_{S(k)})$はユーザー$u$がアイテム$S(k)$をCVする確率である。$\hat{q}$は、ログデータから機械学習モデルを学習させることで得る。

DM推定量は、$\hat{q}$の予測精度に依存したバイアスを持つが、バリアンスが小さい利点がある。

\subsubsection{DR推定量}

\clearpage

\section{検索におけるランキングのオフ方策評価}

\subsection{KPIを設定する}

KPIは、推薦枠において発生する総CV数とする。

\subsection{データの観測構造をモデル化する}

対象となる推薦枠においてコンバージョンが発生するまでの一連の流れを以下のように整理する。

\begin{enumerate}
    \item ユーザーがサービスに訪問する。
    \item ユーザーに検索条件を提示する。
    \item ユーザーが検索条件を選択する。
    \item 訪問ユーザーに対して、推薦枠内に複数のアイテムがランキング形式で表示される。
    \item ユーザーの興味において、推薦枠内のいずれかのアイテムがクリックされる。クリックが発生しないケースもあり得る。
    \item クリックされたアイテムについて、アイテム詳細ページに遷移した後、ユーザーの意思決定に基づきコンバージョンが発生するか否かが決定される。
\end{enumerate}

ログデータ$\mathcal{D}$は、以下のように定義する。
\begin{equation}
    \mathcal{D} = \{(u_{i}, q_{i}, l_{i}, S_{i}, C_{i}, C_{i}R_{i})\}_{i = 1}^{n}
\end{equation}

ログデータ$\mathcal{D}$の各要素は、独立同分布に従うとする。ログデータ$\mathcal{D}$の従う確率分布は、
\begin{equation}
    p(\mathcal{D}) = \prod_{i = 1}^{n} p(u_{i}, q_{i}, l_{i}, S_{i}, C_{i}, C_{i}R_{i}) = \prod_{i = 1}^{n} p(u_{i}) \pi^{q}_{0}(q_{i}|x_{u_{i}}) p(l_{i}|x_{u_{i}}, q_{i}) \pi^{s}_{0}(S_{i} | x_{u_{i}}, q_{i}, l_{i}) p(C_{i}, R_{i} | x_{u_{i}}, q_{i}, l_{i}, S_{i})
\end{equation}
とする。

\subsection{解くべき問題を特定する}

KPIは、推薦枠において発生する総CV数とした。これより、方策の価値は推薦枠における期待CV数を表す
\begin{equation}
    V(\pi) = \mathbb{E}_{p(u)\pi(S|x_{u})p(C, R|x_{u}, S)}\left[\sum_{k = 1}^{K} C(k)R(k)\right]
\end{equation}
とする。

\clearpage

\section{プラットフォーム全体におけるランキングのオフ方策評価}

\end{document}